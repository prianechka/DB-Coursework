\chapter{Аналитический раздел}

\section{Основные определения}
В статье 4 Федерального Закона от 29.12.2006 N 244-ФЗ (ред. от 02.07.2021) "О государственном регулировании деятельности по организации и проведению азартных игр и о внесении изменений в некоторые законодательные акты Российской Федерации" содержатся следующие определения:

\textbf{Азартная игра} -- основанное на риске соглашение о выигрыше, заключенное двумя или несколькими участниками такого соглашения между собой либо с организатором азартной игры по правилам, установленным организатором азартной игры.

\textbf{Пари} -- азартная игра, при которой исход основанного на риске соглашения о выигрыше, заключаемого двумя или несколькими участниками пари между собой либо с организатором данного вида азартной игры, зависит от события, относительно которого неизвестно, наступит оно или нет.

\textbf{Интерактивная ставка} - денежные средства, в том числе электронные денежные средства, передаваемые с использованием электронных средств платежа, в том числе посредством информационно-телекоммуникационных сетей, включая сеть "Интернет".

\textbf{Букмекерская контора} - игорное заведение, в котором организатор азартных игр заключает пари с участниками данного вида азартных игр.

\section{Принцип работы букмекерской конторы}
Букмекерская контора предлагает игрокам заключить пари на различные события. 
Тип событий может быть различным, но почти всегда букмекерская линия представлена спортивными соревнованиями.
Спорт доминирует на рынке, так как является непредсказуемым, матчи происходят каждый день, и по всему миру живёт огромное количество болельщиков.

Букмекерская контора предлагает коэффициенты на возможные исходы события. Например, в случае футбольного матча такими исходами является победа одной из команд или ничья. Игрок имеет возможность заключить с конторой пари на наступление какого-либо исхода, сделав ставку. Если этот исход наступил, то букмекерская контора должна вернуть игроку денежные средства на сумму, умноженную на коэффициент исхода. В противном случае букмекерская контора оставляет сумму ставки у себя.

Рассмотрим первое приближение выбора коэффициента. Пусть $p$ - вероятность наступления какого-либо исхода. Тогда коэффициент рассчитывается по формуле \ref{first}: 
\begin{equation}\label{first}
k = \frac{1}{p}
\end{equation}

Пусть $N$ -- количество ставок игрока, а $S$ -- сумма одной ставки. Математическое ожидание заработка игрока можно посчитать по формуле \ref{win}:
\begin{equation}\label{win} 
M = k_1 * S * p * N - S * N = \frac{1}{p} * S * p * N - S * N = 0
\end{equation}

Так как выигрыш букмекера формируется из проигрыша игрока, из \ref{win} следует, что и математическое ожидание заработка букмекера тоже равняется нулю.

Но букмекерская контора рассчитывает зарабатывать и в краткосрочной, и в долгосрочной перспективе. Для получения прибыли в каждом матче в коэффициенты закладывается маржа. Пусть $p_{win}$ - процент маржи, которые контора хочет иметь, а $\omega$ -- количество исходов события. Тогда коэффициент рассчитывается по формуле \ref{second}:
\begin{equation}\label{second}
k = \frac{1}{p + \frac{p_{win}}{\omega}}
\end{equation}

В таком случае математическое ожидание заработка игрока будет отрицательным, так как коэффициент стал меньше, чем в \ref{first}. Исходя из этого, букмекерская контора остаётся в выигрыша при наступлении любого исхода.

Для того, чтобы постоянно зарабатывать, букмекерам требуется безошибочно рассчитывать вероятности наступления всех событий. 
Этим обычно занимаются профессиональные аналитики и статистики.

Теперь рассмотрим событие, в котором есть два противоположных исхода -- выигрыш одной из команд. Пусть $k_1$ и $k_2$ -- коэффициенты на эти исходы, а $S_1$ и $S_2$ -- денежная сумма, поставленная на эти исходы. 
Если реализуется первый исход, то заработок БК составит $S_1 + S_2 - k_1 * S_1$, а если второй исход -- 
$S_1 + S_2 - k_2 * S_2$. 

Денежные потоки могут быть распределены таким образом, что их распределение не будет соответствовать рассчитанным вероятностям наступления событий. 
В таком случае при наступлении одного из событий БК потеряет деньги.
Следовательно, требуется постоянно обновлять коэффициенты, опираясь на объём средств, поставленных игроками на каждый исход.

При этом всём на букмекерском рынке большая конкуренция, следовательно, требуется предлагать самые выгодные коэффициенты на рынке.

Подводя итог, расчёт коэффициентов на матчи -- сложная математическая, экономическая и маркетинговая задача, которая и позволяет букмекерам зарабатывать деньги.