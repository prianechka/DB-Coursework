\usepackage{mathtext}
\usepackage{amsmath}
\usepackage{amsfonts}
\usepackage{amssymb}
\usepackage[russian]{babel}
\usepackage{latexsym}
\usepackage{indentfirst}
\usepackage{mathtools}
\usepackage{graphicx}
\usepackage{csvsimple}
\graphicspath{}
\DeclareGraphicsExtensions{.pdf,.png,.jpg, .eps, .svg}
\usepackage[top=20mm, right = 10mm, left = 30mm, bottom = 20mm]{geometry}
\linespread{1.3}
\usepackage{color} 
\usepackage{listings} 
\usepackage{pythonhighlight}
\usepackage{caption}
\usepackage{ragged2e}
\justifying


\usepackage{amsmath}

\usepackage{setspace}
\onehalfspacing % Полуторный интервал

\frenchspacing
\usepackage{indentfirst} % Красная строка

\usepackage{titlesec}
\titleformat{\section}
{\normalsize\bfseries}
{\thesection}
{1em}{}
\titlespacing*{\chapter}{0pt}{-30pt}{8pt}
\titlespacing*{\section}{\parindent}{*4}{*4}
\titlespacing*{\subsection}{\parindent}{*4}{*4}

\usepackage{titlesec}
\titleformat{\chapter}{\LARGE\bfseries}{\thechapter}{20pt}{\LARGE\bfseries}
\titleformat{\section}{\Large\bfseries}{\thesection}{20pt}{\Large\bfseries}

\usepackage[figurename=Рисунок]{caption}
\usepackage{placeins}

\usepackage{caption}
\captionsetup{justification=raggedright,singlelinecheck=false}

\usepackage{cmap} % Улучшенный поиск русских слов в полученном pdf-файле
\usepackage{fontspec}
\setmainfont{Times New Roman}